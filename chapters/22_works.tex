\section{Who says that Tantra seminars work?}

\begin{center}
\includegraphics[width=7cm]{images/22_works.png}
\end{center}

\textit{I am saying that. And I can prove it.}

For 23 years now, I have been giving workshops in the area of self development and Tantra, and for 23 years I have been hearing the very same statements over and over again, such as "\textit{I don't need Tantra. I also get enough sex without it.}" or "\textit{Tantra? That's the thing with the orgies, right?}".

These ideas surprise me every time. They have absolutely nothing to do with what I experience daily in groups regarding personal growth, opening the heart, increasing life joy, and at the same time regarding the ability to challenge rigid convictions and to think independently.

But maybe I imagine all of that only? We all like to perceive only those things which we like the most, what we want to be true, and as a Tantra teacher I of course wish that my participants reach their goal.

I wanted to know it exactly, so I started to systematically gather some data. After eleven years of observation and collecting information, I wrote my thesis using my own institute as an example: \footnote{\url{https://independent.academia.edu/HelenaKrivan}}{\textit{Tantra Workshops: A Fresh Take on Ancient Paths to Inner Peace}}. In this paper, I describe the difficulties to define Tantra (many scientists have failed to do so), sketch the goals of Tantra according to the understanding of the Namasté institute, and shed light on what Tantra has to do with sexuality, happiness, inner peace and much more. I use examples of typical exercises to illustrate an actual workshop.

The core of the case study are six profound interviews with three men and three women, all of whom graduated at our yearly training. In those intense, one hour long conversations, it is made visible how the personal view on topics such as relationships (to oneself, family, friends and the world itself), emotions, body perception, serenity or self-esteem has improved.

There is this young woman who decided as a young child to not allow any physical touch anymore and almost died emotionally because of it. She is now able to hug people, allowing intimacy, and has made peace with her once despised body.

There is this young man, who avoided any kind of stress and confrontation his whole life and rather expressed renouncement. Today, he is grounded in his opinion, able to spontaneously speak in front of an audience and perceived as stable and sovereign by his environment.

Another interview partner, who tried to reconcile his shattered family his whole life, has made peace and has let go of his accusations, having found his place in life.

A mother who has not spoken a single word to her daughter for many years was able to establish a trusting relationship with her and even could find carefree joy in life after a serious illness.
A young woman had coped unhealthily with stress for many years by finding escape in computer games and the fridge. She ended up finding new ways to get herself "out of the hole": Gentle, conscious breathing, getting herself support and being there for herself with kindness. All of those things had seemed impossible before.

These astonishing changes are the result of a consequent and carefully built up education in conscious introspection and mindful self-reflection.

The interview partners report that now they live in the moment more often instead of in the past or future. They deal with stress with much more calmness and feel considerably more at ease with and in their bodies than before the one-year training. Also the newly experienced naturalness and refreshing innocence of their sexuality is emphasized often. All three women also pointed out that their view on men has changed after having had the opportunity to see men in their sensitivity, empathy and vulnerability.

Yet the most impressive reports concern very personal moments of spiritual experiences and a super-personal, transcendental love which cannot be described in words. Two of the interviewees have learned how to actively decide what they want to think, meaning they are not only at the mercy of their hamster wheel in their head, but can quickly identify and deliberately turn off destructive thought patterns.

There is yet too little research being done about Western Tantra, which is where Tantra workshops and retreats belong to. I wish from the bottom of my heart that this case study will be an impulse for my fellow colleagues to scientifically substantiate their observations, experiences and results and make them accessible to a wider audience. This way, our common work can unfold its healing, reconciliatory and liberating effect on the individual person and the society.
