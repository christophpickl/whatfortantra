\section{Just married + Tantra}

\begin{center}
\includegraphics[width=11cm]{images/16_married.jpg}
\end{center}

\textit{How do Tantra retreats work when you are married?}

That frequently asked question shows that for many married participants this is a contradiction in itself. And they are right. Although, only if you assume that Tantra means living out pubescent desires for endless sex without commitment, responsibility and connection.

If you think, on the other hand, that first you have to get rid of the skeletons in your closet in order to start and maintain a happy relationship, things start to look very different.

Additionally, if you wisely choose a partner who rarely projects his own childhood traumas onto you and mostly  copes with his emotions, needs, fears and frustrations - just like you -  the chances rise considerably for a love which is "free" in its wider definition. Free not of commitment and responsibility, but free of neurosis and childish projections.

If this is paired with the insight that relationships are a life long learning process, with oneself, with the other, and with each other, that there is no higher level of Yoga than marriage. Then we don't simply throw away everything and look for "something better" just because things escalate every once and then (Heaven can be full of trumpets instead of violins sometimes.). And we realize that our happiness is not our partner's but our very own responsibility. Then the combination of married and participating in tantric retreats for personal development is absolutely plausible. 

In order to experience, explore and understand ourselves as whole human beings, we also have to carefully deal with aspects of sexuality. A small part of Tantra deals with this topic. This exploration can be done in exercises, meditations and reflections, alone or together with others. It is not really surprising that as a couple, one will be faced with fears of not being good enough and abandonment. This contact with one's own personal fears doesn't need to be followed by a jealousy scene: If I can love pretty free of neurosis, then I am also pretty free of the obsessive need to "own" my partner  exclusively, permanently and forever, or at least for how long it suits me.

We accompany many couples who go on an exploration together but also separately, sometimes joining the same retreat, sometimes alternating. It is encouraging what they have to report: They mention unisono the growing closeness, emotional safety, re-established bond and deepened intimacy.

Awakened love does not want to own the other, but to see the other happy. That's why long-term relationships are such an inexhaustible source for tantric practice. Much more than just exhilarating, erotic togetherness.
