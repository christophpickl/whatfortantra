\section{Teacher wanted}

\begin{center}
\includegraphics[width=7cm]{images/07_teacher.jpg}
\end{center}

\textit{How to find a good teacher.}

If someone is now curious about Tantra and wants to explore it, how do they  get started?

Learning it by simply studying a book is not only contradictory to its oral tradition, it is most probably also a waste of time, as most exercises can't be learned through written language. Especially during the first part of a Tantric path, a group is helpful. A group gives me the feeling of safety and supports me in seeing my potential for development. Along with that, having someone available who I trust, rely on and can ask questions helps me when in need in difficult times. As he or she has lots of Tantric experience of their own, this is a priceless support.

To find a proper teacher is a big project. One might think it should be easier these days, as in the past there was less offer and one had to travel very far in order to experience a retreat. Looking up the word "Tantra" in a search engine today, one is confronted with a flood of facilitators. This leaves lots of interested people clueless and they give up with frustration.

The following paragraphs contain a few indications which can but don't have to be regarded when looking for a trustworthy Tantra institution. If one doesn't find one, like me in my beginning, the lessons can also be fruitful, but far less enjoyable and the detours can be quite long.

\begin{itemize}
\item Finding the right teacher is special. Don't expect that it will work right from the beginning at the first attempt. Take your time, have patience. And yes, it may take years.
\item Which educational background does your future teacher have, what kind of qualifications? Is there proof for it? Do you know your teacher's teachers? If not, try to look them up on the internet. Go directly to the root source.
\item What does the internet tell you about your potential teacher? Don't be naive and don’t believe everything. Find facts, and carefully pay attention.
\item Are personal services and favors expected or even demanded, when you are in individual contact with the teacher?
\item Are common rules laid out transparently and are they comprehensible and challengeable?
\item Especially because Tantra works with sexual energy, it is of utmost importance to draw a clear line between teachers (including assistants) and students. Erotic innuendos, offers and requests coming from the teaching staff are a red flag. The same is true when teachers accept erotic offers from the students.
\item How does the teacher deal with his or her own weaknesses, flaws and mistakes? Is she able to admit to not knowing something, to be mistaken, or simply, which is very human, to have messed up something?
\item Does the teacher claim to be enlightened or to possess some special skills? Does he permit that others say similar things about him? What happens if doubt is expressed, or if he is critized in front of others?
\item Have a look at several teachers, within a retreat and also outside, in their "ordinary life". When they write books or articles, read them. Then make up your own mind.
\item Contact them personally. As long as you are not intrusive and stay respectful, a clear, friendly and in-time response to your questions should be possible. A guru figure, which is not even reachable via email, is of little help to students.
\item Have a close look at the teacher’s students, especially the advanced ones.. Are they good role models for you? Do you also want to think, talk and act like them?
\item What is the atmosphere in the group? How do they deal with tension, fears and resistances? Is there laughter, but not about others?
\item Check possibilities to opt out. It should be easily possible, at any time.
\item It is a good sign if your teacher motivates you to also have a look at other teachers.
\end{itemize}

You ticked all the checkboxes above, it looks promising and you have a good feeling about him or her? Then your journey can begin. A lot of joy and insights to you.
