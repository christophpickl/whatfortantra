\section{What is Tantra?}

\begin{center}
\includegraphics[width=7cm]{images/06_what.png}
\end{center}

\textit{The traditional Indian teaching can bring a fresh, erotic breeze into the bedroom, at least according to popular sexual advisors. Only few people know what Tantra is exactly about and what its actual purpose is. One thing is for sure though: Western Tantra can have a positive effect and be healing for every relationship in your life.}

So what is Tantra? This question pops up quite frequently. From people interested in spirituality, retreat participants, buddhistic practitioners and also from journalists.

The answer to that question can vary, as the topic itself is rather intractable, as it twists itself literally like the Kundalini snake and uncompromisingly requires an ever new approach.

Even the translation from the Sanskrit word \textit{Tantra} withdraws from a one-dimensional approach. Sanskrit dictionaries define the meaning of the syllables \textit{Tantra} like this: \textit{remedy, joy, loom, fire test, trick, oath, right approach}, and \textit{cause of more than one effect}. Other important meanings of \textit{Tantra} are \textit{scientific work} or \textit{religious discourse} (in reference to \textit{Tantra} as a designation for treatises from the Tantric Buddishm, such as Kalachakra-Tantra). Whereas in western culture, the word Tantra is understood as an umbrella term of methods for personal development, giving space to the non-hedonistic (not only pleasure focused) sexuality aspect of it.

These multiple meanings of the term often lead to misunderstandings among different Tantra practitioners. Many of the definitions are self-explanatory, yet I personally like the weaving, connecting aspect of the term the most. What exactly is being woven on the tantric loom and connected? Simply put: Opposites. They will be intentionally connected, until they dissolve and the boundaries are interwoven. Until the practitioner has catapulted herself to a space beyond duality and beyond the level of mental comprehension and, at least for some seconds, has reached her goal of pursuit. That's more easily said than done.

The dissolution of opposites, of contradictions, comes with massive inner resistances, as our mind has its difficulties to accept that there is something else beyond hot and cold, good and bad, existent and non-existent and especially that there should be something outside our mind. Tantric approaches always were revolutionary and created lots of resistance. That's also the reason why Tantra never became mainstream and why instead only a few seemingly half-insane people were concerned with it.
The early forms of Tantra arose about 4000 years ago in the south of India in the realms of Hinduism. Tantra radically stood against everything which was sacred to the older generation and which was considered the only true path of enlightenment: Strict purity, a cast system and a rigorous prohibition of meat and alcohol.

The disrespectful young generation claimed, on the other hand, that awakening will be  achieved exactly when worldly temptation will no longer be resisted with self-mortification. To the contrary: The temptation should be acted out fully in a sacral context in order to worship the gods, until its void, its meaninglessness, will simply reveal itself.

That might sound aloof or unreachable, yet something quite similar to that revelation can also be experienced in regular daily life: When, for example, after two hours of consuming social media, I slowly realize how mechanical, self-affirming and empty I feel after losing myself in this temptation.

Whoever wants to deal with unadulterated Tantra needs to bring a considerable amount of courage and lots of desire for change, because traditional Tantra still counts on provocation. Yet never for the sake of provocation itself but always to induce a fundamental change in one's thinking: To reach the realization of reality through temporarily turning off the conventional rules about this reality.
Some of those methods are surprisingly contemporary and psychologically well founded. Today, they can be found in texts such as the Hevajra-Tantra, which are the foundation for Sadhanas (practice texts) of the Anuttara-Yoga-Tantra.

What was valid back then is still valid today: for such an exceptional progress an exceptional amount of energy is necessary. It requires persistence while practicing, determination to continue, courage in difficult times, confidence when nothing seems to work out and the whole self-constructed structure of reality seems to break down. It especially requires wisdom and humility as strengths when at some point results from the practice show up.

But where should this huge amount of energy that is necessary to see reality for what it is come from? Where is that source, which is easily accessible, inexhaustible and holds irresistible powerful bubbles? The answer of the early Tantrics was obvious and still leaves people incredibly amazed : The force of creation. The power of creativity. The life force itself: Sexuality.

Tantra thus merges spirituality and sexuality and brings back the deep, mystical meaning to sexuality. Because what else could be more awe-inspiring than the potential to create new life? And nothing gets lost from this incredible awe-inspiring force if it is being used not for external, but for internal creation. This turbo boost enables us to experience this deep insight in a relatively short time. In contrast to what most usually think, Tantra never sees sexuality as a means of its own end. It is not about pleasure; that's only a comfortable side effect. Sexuality is just a medium, a vehicle, a form of meditation. Practitioners intentionally use it without creating a personal bond, to create enough energy and thus to quicker reach the goal of gaining a deeper understanding.

So if Tantra has nothing to do with sexuality as an expression of love and attachment towards a specific person, how is it related to relationships? Exactly: Not at all. Or at best maybe just as a side effect.

Yet again and contrary to what most people might think, Tantra is no exotic miraculous potion which helps worn-out relationships with Mantra singing, the smell of incense and acrobatic positions. If one understands Tantra in its traditional context, without turning it into  sex-gymnastics or couple therapy, then it becomes a practice for individuals which support each other along their path.
If we follow the hinduistic path, then - in a nutshell - one aims at liberating oneself from illusions. If our Tantra practice leans towards the buddhistic path, then we strive to contribute to the well-being of all beings.

Bearing all of the above in mind, are relationships in Tantra not an issue at all? Well yes, of course, if only a bit different than we are used to. One of the fundamental methods in Tantra is uniting contradictions. This can happen in the ritualistic union between the masculine and the feminine. Or it can be much simpler and ordinary. For example, by making it a daily practice to understand that everything (literally everything!) we perceive is a projection of our own mind, and the separation of me and not-me is just an illusion. No matter who I support, insult, admire or despise: It will always affect myself. That's why the dividing line between me and "the others", whether it might be based on nationality, income or attitudes, is just another illusion which keeps me imprisoned in a cage caused by an insane worldview.

Once I free myself from the chains of emotional and mental entanglement through practice, even just a bit, it will automatically affect all my relationships in a positive way. I will slowly stop blaming others for the discomfort in my life, understand the causes of their own entanglement better, and am more tolerant and patient with them. That would be a meaningful tantric exercise program in addition to a heightened consciousness, body awareness, a bright attitude and gratitude. Many think, however, that they already experienced Tantra by paying for a Tantra massage. Some couples also think that they have a "Tantric relationship", whereas they simply have an open relationship with changing sexual partners. In fact, polyamory only has something to do with Tantra marginally.
Whoever wants long-lasting Tantric joy and light-footedness in life has to invest quite some time and effort upfront. Authentic Tantra is not particularly consumer friendly. Instead it is rather effortful and not flattering for the ego if one is serious about the "get to know oneself" part. And that's exactly why we need these previously mentioned amounts of energy which we can very easily scoop from joyful, erotic activities. The catch here is that the original Tantra assumes that we have a natural, heartfelt and relaxed relationship with sexuality. If we have such a relationship, the bubbling sources of creative energy are truly available to us and we can cheerfully set out to discover all of it. Just who of us has a truly relaxed, neurosis-, stress- and embarrassment-free and joyful relationship with their sexuality? So in order to work with those sources of energy, we have to make them free and available first. That's why the so-called Western Tantra at it's beginner stage often resembles Biodanza and communication training, and at its intermediate stage it seems like body work, therapy and a school of life.

That way we train the body, mind and heart; gently, in stages and interconnectedly. The result is a holistic development of the whole human. Methods which ignore the body or even despise it are - from the point of view of Tantra - unfavorable, as without the body we would have no senses and therefore no sensual impressions. Without those impressions we would not be able to perceive the duality and consequently not transcend it.

