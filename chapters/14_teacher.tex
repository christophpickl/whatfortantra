\section{The only true teacher}

\begin{center}
\includegraphics[width=7cm]{images/14_teacher.png}
\end{center}

We regularly receive emails from people who explicitly ask for a specific facilitator and sometimes add: "If X or Y isn't going to facilitate, then I won't come... Because s/he is so great, and I don't know the others from the team and I'm also not interested in them."

Well, to participate in an event that is very much about getting to know oneself, doing bodywork and also about healing the so often crushed sexuality, definitely requires trust. Without reasonable trust in the facilitator, many would be constantly on guard to protect themselves. This way, the gains from such an event could be rather small.

That's why as a participant, of course you make sure that you can trust those who instruct the exercises and play the music. This trust grows with subsequent events, through direct conservations and questions. Also single practice evenings and other short events are a good possibility to verify and deepen your trust.

Of course you will have your personal preferences among the teachers, just the way I did, at least at the beginning, and just the way I had my preferred exercise partners. But this should not be the final state of the development which we guide in retreats. Because if everything goes as expected, at the end you should be more and more indifferent (German: "\textit{gleich-gültig}", lit. "same-valid") with whom you do the exercises. Then I know, based on all the experience I already gained, that I do the exercises first and foremost for myself, to get to know something about me. My exercise partner is an essential help and therefore very welcome, yet, s/he has less and less influence over time on whether it will be a "good" or a "difficult" exercise for me.

Advanced participants have gloriously reported that they can now do any exercise with literally everyone from the group in a relaxed and joyful way, regardless of whether they are well known or strangers.

A similar situation can be seen with the facilitators: If everything goes to plan, you can slowly let them go. Their roles as substitute parents and projection surfaces that were natural and important at the beginning can now be put behind. Now the responsibility for personal happiness and well-being, your very own safety and the trust in oneself and the world is not passed over to someone "up there", but gradually taken over by oneself. And that's exactly how it is supposed to be.

The goal of a good facilitation team can't be to be adored, applauded to and to be used. To the contrary: The goal should be to put oneself as a facilitator more and more into the background and to become more and more superfluous. Even to a degree  that the participants only vaguely notice that there is someone at the front and instructs, while they relax and confidently sink into the exercises and experience.
