\section{Tantra: Beneficial for the backbone}

\begin{center}
\includegraphics[width=7cm]{images/08_backbone.jpg}
\end{center}

\ldots and with that, not only the bodily, moveable elements of a Tantra retreat are meant, which have astonishing, relaxing effects for many people! When you start to deal with Tantra, you usually have already gotten some life experience, enough in any case to have figured out that not everything on this planet runs smoothly: People get mobbed in the subway, contracts turn out to be a fraud, politicians are corrupt, relationships shatter, intrigues take place in the office and your best friends cheats on you.

Now there are many ways to deal with these depressing facts: One can duck one's head and pretend that nothing is happening and hope it will pass soon. One can shrug one's shoulders, sighing that the world is simply ruined. Or one can stand up and do what needs to be done. That's what's called moral courage, which is not very attractive, as who wants to expose oneself?! To speak up and probably get a slap for it? Who wants to get in trouble with the ones who are in power, like the daring activists have been doing? To do it nevertheless requires a lot of courage and the same amount of wisdom, because without wisdom, courage is just crazy foolhardiness, and wisdom without courage only sits uselessly behind the oven.

And what does all of this have to do with Tantra? Tantra, in the way we understand it, has only peripherally to do with sexuality. It is more concerned with getting to know oneself as well as possible. People who know themselves better estimate their potential more realistically. That's how they are able to keep standing up straight, even if others try to make them small. That requires a spine, and that spine is not given to us at birth, but we have to acquire it through hard work.

And how does that work? By getting out of our comfort zone, into unfamiliar, challenging situations and, with the support of others, master them. Because fear starts to dissolve when we don't avoid it, but kindly approach it and finally embrace it. We gain a backbone not only with our body and mind, but at the same time by training our courage, compassion and serenity. For example with Tantra.
