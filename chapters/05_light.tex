\section{Light in dark rooms}

\begin{center}
\includegraphics[width=7cm]{images/05_light.png}
\end{center}

Some people come to us with the conviction that Tantra has something to do with spirituality. And they are damn right. Yet others are convinced Tantra has only to do with spirituality. And they are damn wrong. Often they already have spent some time on other paths which primarily focus on the spiritual component of being, and encounter Tantra along their way.

With Tantra, for a long time and especially at the very beginning, it is plainly about the body, the simple physical body: its needs which want to be understood and cared for, which has to eat and digest, which not always smells like roses and, with increasing age, shows uncomfortable defects. All of that seems to be contradictory to a spiritual goal: It is about the divine, the pure, the eternal, untouched by the earthly, whereas the material world is full of hardship, desire and sweat and constantly reminds us that everything which has a beginning also has an ending and dies. An insight which the human mind will accept only very reluctantly.

For many it seems wiser to focus on the higher, on the pure energetical. That means they meditate solely on the 6th and 7th chakra, on the sources of the ethereal and "let go of the bodily". And what is tempting about it: It works! At least for some time. One receives astonishing results indeed, visions appear, unforeseen spiritual spaces open up ... one has the feeling of finally having arrived.

If there still wouldn't be that bothersome lump, the body, which still has needs, producing less enlightened excretions, revenge itself with inflammation of the lungs if not protected properly and stubbornly has desire for desires. Many decide at this point to go against the body, denying it its needs until it finally gives up. Tantra goes in another direction.

It presumes that not the rich need financial support, but the poor ones. Not the spiritual needs to be freed - it was and has actually always been freed! - but the materialistically detained needs spiritual attention and transformation. Which refers to the body and everything which has to do with our dark, unbeloved aspects.

What good does it do to shine outside into a bright sunshiny day with a flashlight? How much more benefit does it bring to use this very same flashlight to (en)lighten all the dark, deep, less appreciated rooms of our own house, so that light and clarity find their way in.

It is indeed true that Tantra is focused on spirituality. But: It does not split off the body and its concerns with the material reality. The body is the temple in which the higher states happen, which makes them even possible in the first place. In order to get there, all layers of the bodily being need to be filled with light and recognition.

And that is exactly where Tantra leads you: It helps you to transform your body energetically so that it is not an unbeloved, materialistic appendix of the mind anymore, but an affectionately cultivated temple.

This way, you can go on adventures together with your body and soul, as a whole and undivided human being.
