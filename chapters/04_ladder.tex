\section{The rung ladder}

\begin{center}
\includegraphics[width=2cm]{images/04_ladder.png}
\end{center}

\textit{"Tantra?! I already did that!"}

So you already dared to attend a try-out seminar and, chapeau, that took quite some courage! One does not know upfront what to expect, what kind of people will be there, whether the facilitators will be strange or guru-like, and of course: What about all these exercises? You have never ever done so much eye-gazing with strangers before, particularly not  at such an intensity. You have also never breathed so intensively, or at least not with such a firework-effect.

At the end you feel touched, showing your gratitude and telling everyone how great it was. Now you can check off Tantra in your internal to do list. Because now you know how it works, right?! And now you are able to contribute to the conversation when people talk about it.

Others from your seminar group, and maybe at some point also you, have the feeling that there must be more and continue. For example, you book a basic retreat. And wow! Although there are many things you recognize from the try-out, it is not comparable at all: You are more familiar with things, know your way around and feel home and cosy in the retreat setting. The simplest exercises have a more intense effect. You start to recognize your first patterns within yourself and others, which obviously make life more difficult, and receive tools to deal with them properly. At the end, everyone expresses their gratitude exuberantly.

Some of us know now exactly that they want to continue their search, and also others know with the same level of confidence that they received exactly as much as they wanted, "more is not needed". Their environment already asks them why they are somehow different, somehow in a better mood. Some of those who want to continue, to get to the bottom of things, consider a full year training. Yet others book the try-out event two, four or even five times  and report that it is new and fascinating each time.

A full year training tackles a bigger construction site: Here you look gently at old patterns, deepen your trust in yourself and the world, make peace with your body and make one or the other experience which you can't really express with words.

After that one year training you finally can relax, lean back, and know that you have mastered Tantra. What else should be there? Some of your seminar group might have a hunch that there must be more, that the actual essence is still ahead. The one year training might have been only something like primary school. And they continue... Those people, supported by the subsequent retreats for advanced participants, report their intention more broadly:

It is not only about the joy of life itself anymore, not about being happy, but about the ability to shape one's own life. It is about goals which go way beyond the ones of daily life and desires. It is more about how to live a life without irrational fears or how to see the essence of things.

Some will breathe lightly after the advanced training and be absolutely certain now they have learned everything that there is to be learned.

And the others? They will unswervingly continue to practice, in the 24/7 retreat of their regular, daily life.