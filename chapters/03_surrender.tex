\section{Surrendering - How does that work?}

\begin{center}
\includegraphics[width=6cm]{images/03_surrender.png}
\end{center}

For many people, it is still a test of courage to book a Tantra retreat: What happens there? What is expected from me? What if someone I know findsout about it? And even worse, what if someone I know will be there?!

Actually going there is already the first test of courage, and then taking the plunge yet another one. But what does it mean precisely, "to surrender" to such an experience?

It means to really make use of the offer (see the "buffet" analogy from the previous chapter), as it is indeed possible to visit a seminar for personal development without developing personally.

For example, I remember a psychologist, who literally sat the whole week at the edge of the scene. In each sharing session, he told everyone what "their topic" was. Or couples which do all the exercises exclusively with each other, sometimes adjusting them, and then feel puzzled why all the others report about all their new experiences and insights. Or participants who receive a call from a friend, their company or a parent right before an important exercise and only return to the seminar room after the exercise.

As you can see, it is easily possible to shoot oneself in the foot and deceive oneself in such an elegant way. Then one can avoid to be confronted with one's own fears, which only the others will be able to see.

The alternative, on the other hand, requires determination and courage: I need to want to experience who I really am, and how my inner world works, in order to overcome my inner resistances piece by piece. I need to be able to look the monster underneath my bed in its eyes, in order to be able to dispose of the harmless dust. I need to be vigorous to not simply stay stuck, although it might seem more comfortable at times, and to not give up on myself, even if I don’t know where to get the energy from to continue. I also must be ready to overcome too much pride and accept support and to realize when I am in need of it.

The good news: If you are consequently chasing after yourself, you will also find yourself. It even doesn't take too long! And what you will find is breathtaking and much bigger and more colorful than everything you were holding onto so tightly until then. If you were courageous enough to already participate in a Tantra retreat: Don't just put your little toe into the water! Jump in fully, with joy, and trust that the water will carry you. Because that thing you jump into is nothing unfamiliar at all.

That water is you.
