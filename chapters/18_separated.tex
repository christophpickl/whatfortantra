\section{The illusion of being separated}

\begin{center}
\includegraphics[width=4cm]{images/18_separated.png}
\end{center}

(\textit{or in German: "ge-trennt-sein", as in "to-cut-being"})

At the very beginning we know exactly: We are the world. The world is us. And the world is dark, warm, chuckles quietly and swings us gently.

As soon as we are roughly six months old, we slowly start to understand: That's me, and that's the others, and around us is the world. And we are proud of it, and so is our environment.

We eagerly develop and grow with that very same eagerness into this perspective, that on the one side there is us, and on the other the rest of the world. By that time, the rest of the world expresses itself with a considerable amount of unkindness, by taking away our teddy bear and not giving us the bike of our dreams. And out of incomprehensible reasons (too cold?! WTF?!) it does not allow us to get some ice cream, but expects us to do our homework.

We grow older and realize: The others are different. Some walk around veiled, some exclusively eat organic whole seed bread for lunch. Others claim girls can't do math, and yet others tell on their presumably best friend.

Fortunately, we know exactly what's the right thing to do, because we learned that from mum and dad. Accordingly, we know that anything that diverts from this is wrong. And our thin line of separation becomes a huge gap. Which also has its advantages: As we have absolutely nothing to do with the outside world, we can perfectly blame it for all the misery that happens to us.

For example, that annoying neighbor's dog, which barks loudly and endlessly, and is the reason why we can't study and already failed our exam for the fifth time now. Or that stupid cop, who takes away our newly gained driver license, only because we drank just a little bit of alcohol. Or that bitch from across the street, who took away our guy. Otherwise, he surely would have stayed with us forever. Or the government, the weather, the school system, the moms, who are to blame for everything anyway.
So it's not really that uncomfortable on this side of the gap between us and the world. Unfortunately, those who are to blame now have the power in their hands, so we are left with a lifelong, outraging and resigning nagging about our helplessness in this hopeless, separated world. And then there is this thing called Tantra.

Tantra tells us unconscionable things like: \textit{Look around. Everything you see is you. That's your world, your universe, created by yourself, shaped by yourself. Anywhere you look - everything is you!}

If you do not immediately fall unconscious due to this vile insinuation (\textit{What, even the wars? The stubborn non-vegans? The stranded whales and cut down rainforests, the doubtful politicians, my terrible job and that neighbor's noisy lawn mower?! Everything is me?}), you are invited to try it out: Look around and think about everything you see in the following way: "\textit{That's me, and that also...}". No, don't just try it soon-ish. Do it now! Exceptions are only valid for confirmed coma patients.

Whatever annoys me, irritates me, makes me sick or angry: It is part of me. Whatever I groundlessly admire: It is part of me. And that's why I can't (ha!) be generous, mild, kind or patient … enough, because I'm all of it for myself!

As it is suddenly no one else's fault, I can start to take things in my own hands... and the gap between us and the world starts to dissolve, until we finally end up where we originated: We are the world.

